\section{Tracking}
\begin{itemize}
  \item new implementation for fieldmap computation: hand over $\Delta_{x,y,z}$ and compute $h$ as the diagonal of the cuboid
  \item implementation creates entire 3D fieldmap
  \item convergence study for time integrator uses same fieldmap throughout
  \item integrator uses different number of steps in each simulation, thus only approximation of error possible
\end{itemize}

\begin{center}
\begin{figure}[p]
  \begin{tikzpicture}

\begin{axis}[
  xmode=log,
  ymode=log,
  xlabel={$\ln(h)$},
  ylabel={relative error},
  legend pos={north west}
  ]
\addplot [color=brewerred]
  table[row sep=crcr]{%
-1.1254563760536	-1.38562253725557\\
-1.47417123476753	-2.72794276976295\\
-1.82181890788414	-3.53164020981824\\
-2.16893040606384	-4.92096086990885\\
};
\addlegendentry{error}
\addplot [color=brewergrey]
  table[row sep=crcr]{%
-1.1254563760536	-1.38562253725557\\
-2.16893040606384	-4.51604462728631\\
};
\addlegendentry{$\ln(h^3)$}
\end{axis}
\end{tikzpicture}%

  \caption{convergence of particle trajectory while refining the grid}
\end{figure}
\end{center}

\begin{center}
\begin{figure}[p]
  \begin{tikzpicture}

\begin{axis}[
  xlabel={z},
  x unit = m,
  ylabel={x},
  y unit = mm,
  zlabel={y},
  z unit=mm,
    legend style={at={(0.5,-0.2)},anchor=north, fill=none}]
  ]

\addplot3[color=brewerblue] table[row sep=crcr]{%
x	y	z\\
0.0095709	-12.926	-42.054\\
0.013753	-11.64	-38.005\\
0.028022	-8.2377	-32.706\\
0.061385	-3.1883	-41.284\\
0.14211	3.9849	-35.13\\
0.25973	9.9172	-39.922\\
0.37902	15.71	-46.065\\
0.49831	21.504	-52.212\\
0.61759	27.297	-58.361\\
0.73688	33.09	-64.51\\
0.85617	38.883	-70.658\\
0.97545	44.676	-76.807\\
1.0947	50.47	-82.956\\
1.214	56.263	-89.104\\
1.3333	62.056	-95.253\\
1.4526	67.849	-101.4\\
1.5719	73.643	-107.55\\
1.6912	79.436	-113.7\\
1.8105	85.229	-119.85\\
1.9297	91.022	-126\\
};
\addlegendentry{$h=0.3245$}

\addplot3[color=brewergreen] table[row sep=crcr]{%
x	y	z\\
0.0095709	-12.926	-42.054\\
0.030312	-10.882	-35.182\\
0.12485	-5.8684	-18.452\\
0.24428	-3.6167	-10.622\\
0.36413	-1.5068	-3.2441\\
0.48398	0.60337	4.1343\\
0.60384	2.7139	11.514\\
0.72369	4.8242	18.894\\
0.84354	6.9342	26.272\\
0.9634	9.0444	33.651\\
1.0833	11.155	41.03\\
1.2031	13.265	48.409\\
1.323	15.375	55.788\\
1.4428	17.485	63.166\\
1.5627	19.595	70.545\\
1.6825	21.705	77.924\\
1.8024	23.816	85.303\\
1.9222	25.926	92.682\\
};
\addlegendentry{$h=0.22897$}

\addplot3[color=brewergrey] table[row sep=crcr]{%
x	y	z\\
0.0095709	-12.926	-42.054\\
0.034767	-9.6274	-31.32\\
0.13907	-3.9953	-13.031\\
0.26231	-1.2588	-4.1888\\
0.38574	1.4392	4.5272\\
0.50916	4.1369	13.242\\
0.63259	6.8359	21.961\\
0.75601	9.5341	30.678\\
0.87944	12.232	39.394\\
1.0029	14.93	48.11\\
1.1263	17.628	56.826\\
1.2497	20.326	65.542\\
1.3731	23.025	74.258\\
1.4966	25.723	82.975\\
1.62	28.421	91.691\\
1.7434	31.119	100.41\\
1.8668	33.817	109.12\\
1.9903	36.515	117.84\\
};
\addlegendentry{$h=0.16173$}

\addplot3[color=brewerbrown] table[row sep=crcr]{%
x	y	z\\
0.0095709	-12.926	-42.054\\
0.049829	-8.0561	-26.209\\
0.1664	-2.6183	-8.5139\\
0.29178	1.0829	3.5342\\
0.41722	4.7781	15.563\\
0.54266	8.4729	27.59\\
0.6681	12.169	39.622\\
0.79353	15.866	51.655\\
0.91897	19.562	63.688\\
1.0444	23.259	75.72\\
1.1698	26.955	87.753\\
1.2953	30.652	99.786\\
1.4207	34.348	111.82\\
1.5462	38.045	123.85\\
1.6716	41.741	135.88\\
1.797	45.438	147.92\\
1.9225	49.134	159.95\\
};
\addlegendentry{$h=0.1143$}

\addplot3[color=brewerred] table[row sep=crcr]{%
x	y	z\\
0.0095709	-12.926	-42.054\\
0.04492	-8.5607	-27.851\\
0.15904	-2.8166	-9.1632\\
0.28425	0.89815	2.922\\
0.40955	4.6042	14.979\\
0.53484	8.3097	27.034\\
0.66013	12.017	39.095\\
0.78542	15.724	51.156\\
0.91071	19.432	63.218\\
1.036	23.139	75.279\\
1.1613	26.847	87.34\\
1.2866	30.554	99.401\\
1.4119	34.261	111.46\\
1.5372	37.969	123.52\\
1.6625	41.676	135.58\\
1.7877	45.383	147.65\\
1.913	49.091	159.71\\
};
\addlegendentry{$h=0.0808$}

\end{axis}
\end{tikzpicture}%

  \caption{particle trajectories during grid refinement, $h$ in [m]}
\end{figure}
\end{center}

\begin{center}
\begin{figure}[p]
  \begin{tikzpicture}

\begin{axis}[
  xlabel={$\ln(h)$},
  ylabel={relative error},
  legend pos={north west}
  ]

\addplot [color=brewerred]
  table[row sep=crcr]{%
0	-3.18451256122981\\
-0.693147180559945	-4.65856545466554\\
-1.38629436111989	-6.33968205239022\\
-2.07944154167984	-7.68340785034293\\
-2.77258872223978	-8.93384651088196\\
-3.46573590279973	-9.89333706080527\\
};
\addlegendentry{error}

\addplot [color=brewergrey]
  table[row sep=crcr]{%
0	-3.18451256122981\\
-3.46573590279973	-10.1159843668293\\
};
\addlegendentry{$\ln(h^2)$}
\end{axis}
\end{tikzpicture}%

  \caption{convergence of particle trajectory while decreasing the time step size}
\end{figure}
\end{center}

\begin{center}
\begin{figure}[p]
  \begin{tikzpicture}

  \begin{axis}[
    xlabel={z},
    x unit = m,
    ylabel={x},
    y unit = mm,
    zlabel={y},
    z unit=mm,
      legend style={at={(0.5,-0.2)},anchor=north, fill=none}]
    ]

\addplot3[color=brewerblue] table[row sep=crcr]{%
x	y	z\\
0.0095709	-12.926	-42.054\\
0.04379	-9.7608	-31.38\\
0.17148	-4.8577	-14.679\\
0.3158	-2.7391	-6.7837\\
0.46016	-0.63254	1.0747\\
0.60452	1.4746	8.9353\\
0.74888	3.5814	16.795\\
0.89323	5.6881	24.654\\
1.0376	7.795	32.514\\
1.1819	9.9019	40.373\\
1.3263	12.009	48.233\\
1.4707	14.116	56.093\\
1.615	16.223	63.952\\
1.7594	18.329	71.812\\
};
\addlegendentry{H=1}

\addplot3[color=brewergreen] table[row sep=crcr]{%
x	y	z\\
0.0095709	-12.926	-42.054\\
0.045626	-9.6219	-30.915\\
0.18191	-4.8109	-14.669\\
0.33037	-2.2	-5.427\\
0.47887	0.40166	3.786\\
0.62737	3.0038	13.002\\
0.77587	5.6055	22.215\\
0.92437	8.2072	31.429\\
1.0729	10.809	40.643\\
1.2214	13.411	49.857\\
1.3699	16.012	59.071\\
1.5184	18.614	68.285\\
1.6669	21.216	77.499\\
1.8154	23.818	86.713\\
};
\addlegendentry{H=0.5}

\addplot3[color=brewergrey] table[row sep=crcr]{%
x	y	z\\
0.0095709	-12.926	-42.054\\
0.046067	-9.5975	-30.843\\
0.18387	-4.7293	-14.439\\
0.33342	-2.1453	-5.3212\\
0.48301	0.43061	3.7705\\
0.63259	3.007	12.864\\
0.78217	5.5829	21.956\\
0.93175	8.1588	31.049\\
1.0813	10.735	40.141\\
1.2309	13.311	49.234\\
1.3805	15.887	58.326\\
1.5301	18.463	67.419\\
1.6797	21.039	76.511\\
1.8292	23.615	85.604\\
};
\addlegendentry{H=0.25}

\addplot3[color=brewerbrown] table[row sep=crcr]{%
x	y	z\\
0.0095709	-12.926	-42.054\\
0.046113	-9.5944	-30.833\\
0.18431	-4.6973	-14.384\\
0.33403	-2.0854	-5.2332\\
0.48378	0.51793	3.8901\\
0.63353	3.1217	13.016\\
0.78328	5.7251	22.139\\
0.93303	8.3284	31.264\\
1.0828	10.932	40.388\\
1.2325	13.535	49.512\\
1.3823	16.139	58.636\\
1.532	18.742	67.76\\
1.6818	21.345	76.884\\
1.8315	23.949	86.008\\
};
\addlegendentry{H=0.125}

\addplot3[color=breweryellow] table[row sep=crcr]{%
x	y	z\\
0.0095709	-12.926	-42.054\\
0.046116	-9.5942	-30.832\\
0.1844	-4.6779	-14.337\\
0.33421	-2.027	-5.0805\\
0.48405	0.61533	4.1492\\
0.63389	3.2582	13.381\\
0.78373	5.9006	22.611\\
0.93357	8.5431	31.842\\
1.0834	11.186	41.072\\
1.2333	13.828	50.303\\
1.3831	16.47	59.533\\
1.5329	19.113	68.764\\
1.6828	21.755	77.994\\
1.8326	24.398	87.225\\
1.9825	27.04	96.456\\
};
\addlegendentry{H=0.0625}

\addplot3[color=brewerpurple] table[row sep=crcr]{%
x	y	z\\
0.0095709	-12.926	-42.054\\
0.046117	-9.5941	-30.832\\
0.18438	-4.6805	-14.338\\
0.33416	-2.0343	-5.087\\
0.48397	0.60335	4.1368\\
0.63379	3.2415	13.363\\
0.7836	5.8792	22.587\\
0.93341	8.517	31.812\\
1.0832	11.155	41.037\\
1.233	13.792	50.261\\
1.3829	16.43	59.486\\
1.5327	19.068	68.71\\
1.6825	21.706	77.935\\
1.8323	24.344	87.16\\
1.9821	26.981	96.384\\
};
\addlegendentry{H=0.03125}

\addplot3[color=brewerred] table[row sep=crcr]{%
x	y	z\\
0.0095709	-12.926	-42.054\\
0.046117	-9.5941	-30.832\\
0.18438	-4.68	-14.337\\
0.33417	-2.0336	-5.086\\
0.48399	0.60438	4.1379\\
0.63381	3.2429	13.364\\
0.78363	5.8809	22.589\\
0.93345	8.5189	31.813\\
1.0833	11.157	41.038\\
1.2331	13.795	50.263\\
1.3829	16.433	59.488\\
1.5327	19.071	68.713\\
1.6826	21.709	77.937\\
1.8324	24.347	87.162\\
1.9822	26.985	96.387\\
};
\addlegendentry{H=0.015625}

\end{axis}
\end{tikzpicture}%

    \caption{particle trajectories during integrator study, H in [ns]}
\end{figure}
\end{center}

\begin{center}
\begin{figure}[H]
  \begin{tikzpicture}
\begin{axis}[
  xlabel = {z},
  x unit = m,
  ylabel = {$p_z$},
  y unit = MeV/c
  ]

\addplot [color=brewerred] table[x, y]{figures/tracking/momentum.dat};

\end{axis}
\end{tikzpicture}

    \caption{longitudinal momentum of particle during simulation}
\end{figure}
\end{center}

\begin{itemize}
  \item emission is handled on my side, either uniformly or normally distributed particles
  \item both types depend on multiple parameters: total charge $Q$, number of particles $N_{prt}$, position of probe particles and also bounds for uniform or $\mu, \sigma$ for normal distribution
  \item emission is performed based on the rotated 2D cathode
\end{itemize}
