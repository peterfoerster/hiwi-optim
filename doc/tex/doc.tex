\subsection{Geometry}
The two latest geometries are depicted in fig.~\ref{fig:geometry_v5} and fig.~\ref{fig:geometry_v6}. The numbers refer to the individual patches in the context of IGA. The patch boundaries are indicated by grey lines.

We also observe triple points within the computational domain. These are defined by material boundaries between air/vacuum, steel and the insulator.

\begin{center}
\begin{figure}[H]
   \begin{subfigure}{0.45\textwidth}
      \includegraphics[width=\textwidth]{fig/geometry_v5}
      \caption{Version 5.}
      \label{fig:geometry_v5}
   \end{subfigure}
   \begin{subfigure}{0.45\textwidth}
      \includegraphics[width=\textwidth]{fig/geometry_v6}
      \caption{Version 6.}
      \label{fig:geometry_v6}
   \end{subfigure}
   % \caption{.}
\end{figure}
\end{center}

% \begin{center}
%    \begin{figure}[p]
%       \begin{tikzpicture}

\begin{axis}[
  enlargelimits=true,
  colormap/YlOrRd,
  point meta min = 0,
  point meta max = 5,
  x unit=m,
  y unit=m,
  legend pos=outer north east]

  \addplot[surf, shader=interp] table[point meta=\thisrow{c}]{figures/60kV/geometry/geometry_1.dat};

  \addplot[surf, shader=interp] table[point meta=\thisrow{c}]{figures/60kV/geometry/geometry_2.dat};

  \addplot[surf, shader=interp] table[point meta=\thisrow{c}]{figures/60kV/geometry/geometry_3.dat};

  \addplot[surf, shader=interp] table[point meta=\thisrow{c}]{figures/60kV/geometry/geometry_4.dat};

  \addplot[surf, shader=interp] table[point meta=\thisrow{c}]{figures/60kV/geometry/geometry_5.dat};

  \addplot[surf, shader=interp] table[point meta=\thisrow{c}]{figures/60kV/geometry/geometry_6.dat};

  \addplot[surf, shader=interp] table[point meta=\thisrow{c}]{figures/60kV/geometry/geometry_7.dat};

  \addplot[surf, shader=interp] table[point meta=\thisrow{c}]{figures/60kV/geometry/geometry_8.dat};

  \addplot[surf, shader=interp] table[point meta=\thisrow{c}]{figures/60kV/geometry/geometry_9.dat};

  \addplot[surf, shader=interp] table[point meta=\thisrow{c}]{figures/60kV/geometry/geometry_10.dat};

  \addplot[surf, shader=interp] table[point meta=\thisrow{c}]{figures/60kV/geometry/geometry_11.dat};

  \addplot[surf, shader=interp] table[point meta=\thisrow{c}]{figures/60kV/geometry/geometry_12.dat};

  \addplot[surf, shader=interp] table[point meta=\thisrow{c}]{figures/60kV/geometry/geometry_13.dat};

  /tikz/font=\normalfont\tiny
  % add patch indices
  \addplot[only marks, point meta=explicit symbolic, color=black, nodes near coords] coordinates{
  (0.14,-0.001) [(1)]
  (0.14,0.013) [(2)]
  (0.14,0.05) [(3)]
  (0.12,0.1) [(4)]
  (0.04,0.1) [(5)]
  (-0.025,0.06) [(6)]
  (-0.025,0.03) [(7)]
  (-0.021,0.02) [(8)]
  (-0.022,0.011) [(9)]
  (-0.025,0) [(10)]
  (0.016,0.012) [(11)]
  (0.027,0.017) [(12)]
  (0.055,0.021) [(13)]
  };

  % add patch boundaries
  \addplot[color=brewergreen, line width=1pt] table{figures/60kV/boundary/boundaries11.dat};
  \addplot[color=brewergrey] table{figures/60kV/boundary/boundaries12.dat};
  \addplot[color=brewerblue, line width=1pt] table{figures/60kV/boundary/boundaries13.dat};
  \addplot[color=brewerred, line width=1pt] table{figures/60kV/boundary/boundaries14.dat};

  \addplot[color=brewergrey] table{figures/60kV/boundary/boundaries21.dat};
  \addplot[color=brewergrey] table{figures/60kV/boundary/boundaries22.dat};
  \addplot[color=brewerblue, line width=1pt] table{figures/60kV/boundary/boundaries23.dat};
  \addplot[color=brewerred, line width=1pt] table{figures/60kV/boundary/boundaries24.dat};

  \addplot[color=brewergrey] table{figures/60kV/boundary/boundaries31.dat};
  \addplot[color=brewergrey] table{figures/60kV/boundary/boundaries32.dat};
  \addplot[color=brewerblue, line width=1pt] table{figures/60kV/boundary/boundaries33.dat};
  \addplot[color=brewerred, line width=1pt] table{figures/60kV/boundary/boundaries34.dat};

  \addplot[color=brewerblue, line width=1pt] table{figures/60kV/boundary/boundaries41.dat};
  \addplot[color=brewerred, line width=1pt] table{figures/60kV/boundary/boundaries42.dat};
  \addplot[color=brewergrey] table{figures/60kV/boundary/boundaries43.dat};
  \addplot[color=brewergrey] table{figures/60kV/boundary/boundaries44.dat};

  \addplot[color=brewerblue, line width=1pt] table{figures/60kV/boundary/boundaries51.dat};
  \addplot[color=brewerred, line width=1pt] table{figures/60kV/boundary/boundaries52.dat};
  \addplot[color=brewergrey] table{figures/60kV/boundary/boundaries53.dat};
  \addplot[color=brewergrey] table{figures/60kV/boundary/boundaries54.dat};

  \addplot[color=brewergrey] table{figures/60kV/boundary/boundaries61.dat};
  \addplot[color=brewergrey] table{figures/60kV/boundary/boundaries62.dat};
  \addplot[color=brewerred, line width=1pt] table{figures/60kV/boundary/boundaries63.dat};
  \addplot[color=brewerblue, line width=1pt] table{figures/60kV/boundary/boundaries64.dat};

  \addplot[color=brewergrey] table{figures/60kV/boundary/boundaries71.dat};
  \addplot[color=brewergrey] table{figures/60kV/boundary/boundaries72.dat};
  \addplot[color=brewerred, line width=1pt] table{figures/60kV/boundary/boundaries73.dat};
  \addplot[color=brewerblue, line width=1pt] table{figures/60kV/boundary/boundaries74.dat};

  \addplot[color=brewergrey] table{figures/60kV/boundary/boundaries81.dat};
  \addplot[color=brewergrey] table{figures/60kV/boundary/boundaries82.dat};
  \addplot[color=brewerred, line width=1pt] table{figures/60kV/boundary/boundaries83.dat};
  \addplot[color=brewerblue, line width=1pt] table{figures/60kV/boundary/boundaries84.dat};

  \addplot[color=brewergrey] table{figures/60kV/boundary/boundaries91.dat};
  \addplot[color=brewergrey] table{figures/60kV/boundary/boundaries92.dat};
  \addplot[color=brewerred, line width=1pt] table{figures/60kV/boundary/boundaries93.dat};
  \addplot[color=brewergrey] table{figures/60kV/boundary/boundaries94.dat};

  \addplot[color=brewergreen, line width=1pt] table{figures/60kV/boundary/boundaries101.dat};
  \addplot[color=brewergrey] table{figures/60kV/boundary/boundaries102.dat};
  \addplot[color=brewerred, line width=1pt] table{figures/60kV/boundary/boundaries103.dat};
  \addplot[color=brewerblue, line width=1pt] table{figures/60kV/boundary/boundaries104.dat};

  \addplot[color=brewerblue, line width=1pt] table{figures/60kV/boundary/boundaries111.dat};
  \addplot[color=brewerblue, line width=1pt] table{figures/60kV/boundary/boundaries112.dat};
  \addplot[color=brewergrey] table{figures/60kV/boundary/boundaries113.dat};
  \addplot[color=brewergrey] table{figures/60kV/boundary/boundaries114.dat};

  \addplot[color=brewergrey] table{figures/60kV/boundary/boundaries121.dat};
  \addplot[color=brewergrey] table{figures/60kV/boundary/boundaries122.dat};
  \addplot[color=brewerblue, line width=1pt] table{figures/60kV/boundary/boundaries123.dat};
  \addplot[color=brewerblue, line width=1pt] table{figures/60kV/boundary/boundaries124.dat};

  \addplot[color=brewerblue, line width=1pt] table{figures/60kV/boundary/boundaries131.dat};
  \addplot[color=brewerblue, line width=1pt] table{figures/60kV/boundary/boundaries132.dat};
  \addplot[color=brewergrey] table{figures/60kV/boundary/boundaries133.dat};
  \addplot[color=brewerblue, line width=1pt] table{figures/60kV/boundary/boundaries134.dat};

\end{axis}
\end{tikzpicture}

%    \end{figure}
% \end{center}
%
% \begin{center}
%    \begin{figure}[p]
%       \begin{tikzpicture}

\begin{axis}[
   scale only axis=true,
   width=0.45\textwidth,
   axis equal,
   try min ticks=4,
   max space between ticks=1000pt,
   enlargelimits=true,
   x unit=m,
   y unit=m,
   legend entries={$C^1$-NURBS, Control Points, Control Polygon},
   legend style={at={(0.75, 0.75)}, anchor=north west, draw=none},
   hide axis
   ]

   \addplot[color=black, ultra thick] table {fig/anodering-init/nurbs_3_2.dat};
   \addplot[color=TUDa-9a, mark=*, only marks, mark size=6] table {fig/anodering-init/nurbs_3_2_coefs.dat};
   \addplot[color=TUDa-0b, dashed, ultra thick] table {fig/anodering-init/nurbs_3_2_net.dat};

   \addplot[color=black, ultra thick] table {fig/anodering-init/nurbs_4_2.dat};
   \addplot[color=TUDa-9a, mark=*, only marks, mark size=6] table {fig/anodering-init/nurbs_4_2_coefs.dat};
   \addplot[color=TUDa-0b, dashed, ultra thick] table {fig/anodering-init/nurbs_4_2_net.dat};

   \addplot[color=black, ultra thick] table {fig/anodering-init/nurbs_7_4.dat};
   \addplot[color=TUDa-9a, mark=*, only marks, mark size=6] table {fig/anodering-init/nurbs_7_4_coefs.dat};
   \addplot[color=TUDa-0b, dashed, ultra thick] table {fig/anodering-init/nurbs_7_4_net.dat};

   \addplot[color=black, ultra thick] table {fig/anodering-init/nurbs_8_4.dat};
   \addplot[color=TUDa-9a, mark=*, only marks, mark size=6] table {fig/anodering-init/nurbs_8_4_coefs.dat};
   \addplot[color=TUDa-0b, dashed, ultra thick] table {fig/anodering-init/nurbs_8_4_net.dat};

   \addplot[color=black, ultra thick] table {fig/anodering-init/nurbs_9_4.dat};
   \addplot[color=TUDa-9a, mark=*, only marks, mark size=6] table {fig/anodering-init/nurbs_9_4_coefs.dat};
   \addplot[color=TUDa-0b, dashed, ultra thick] table {fig/anodering-init/nurbs_9_4_net.dat};

   \addplot[color=black, ultra thick] table {fig/anodering-init/nurbs_10_4.dat};
   \addplot[color=TUDa-9a, mark=*, only marks, mark size=6] table {fig/anodering-init/nurbs_10_4_coefs.dat};
   \addplot[color=TUDa-0b, dashed, ultra thick] table {fig/anodering-init/nurbs_10_4_net.dat};

   \addplot[color=black, ultra thick] table {fig/anodering-init/nurbs_13_1.dat};
   \addplot[color=TUDa-9a, mark=*, only marks, mark size=6] table {fig/anodering-init/nurbs_13_1_coefs.dat};
   \addplot[color=TUDa-0b, dashed, ultra thick] table {fig/anodering-init/nurbs_13_1_net.dat};

\end{axis}

\end{tikzpicture}

%    \end{figure}
% \end{center}

\subsection{Electric Field}
The solution for the electric field are shown in fig.~\ref{fig:200kV_v6} and fig.~\ref{fig:300kV_v6}.
It was computed with $p=2$ as the degree of the basis functions and $n_\mathrm{sub}=16$ as the number of elements that each knot vector is uniformly split into.
It is evident that there exist parts of the domain where the magnitude of the electric field is close to the limit of $10 \frac{\mathrm{MV}}{\mathrm{m}}$. There are also very high gradients visible at the triple points, however these also coincide with sharp corners so numerical issues might play a role.

\begin{center}
\begin{figure}[H]
   \begin{subfigure}{0.45\textwidth}
      \includegraphics[width=\textwidth]{fig/200kV_v6}
      \caption{200kV.}
      \label{fig:200kV_v6}
   \end{subfigure}
   \begin{subfigure}{0.45\textwidth}
      \includegraphics[width=\textwidth]{fig/300kV_v6}
      \caption{300kV.}
      \label{fig:300kV_v6}
   \end{subfigure}
\end{figure}
\end{center}

\subsection{Optimization}
The aim of the optimization is to minimize the maximal field amplitudes. For practicality we will only look at the critical points of the geometry, i.~e.~the curvatures of the upper electrode and their respective control points are the DoFs. Aside from geometrical constraints we also consider a volume constraint.

\subsubsection{Cost Function}

\subsubsection{Volume Constraint}

\subsubsection{Additional Constraints}
As an additional constraint we require $C^1$ continuity of the section that is optimized. The starting nurbs is depicted in fig.~\ref{fig:nurbs}. Furthermore the control points, which represent the DoFs, are constrained to stay inside their respective patch boundaries and in the correct order.

\begin{center}
\begin{figure}[H]
   % \input{}
   \caption{$C^1$ continuous nurbs that is to be optimized including the variable control points which represent the DoFs.}
   \label{fig:nurbs}
\end{figure}
\end{center}

\subsubsection{NLopt}
For the optimization we used implementations of BOBYQA and COBYLA as given by the NLopt libary.

\subsubsection{Results}

\begin{center}
\begin{figure}[H]
   % \includegraphics[width=\textwidth]{}
   \caption{Optimization result using BOBYQA.}
   \label{fig:results}
\end{figure}
\end{center}

In order to compare the results with the original geometry and simulation the optimized geometry was also imported and simulated in CST. The results indicate a clear improvement from the optimized geometry over the original one. This is in agreement with the numerical improvement observed in the context of IGA.

\begin{center}
\begin{figure}[H]
   \begin{subfigure}{0.45\textwidth}
      % \includegraphics[width=\textwidth]{}
   \end{subfigure}
   \begin{subfigure}{0.45\textwidth}
      % \includegraphics[width=\textwidth]{}
   \end{subfigure}
   \caption{Mesh of original and optimized geometry in CST.}
\end{figure}
\end{center}

\begin{center}
\begin{figure}[H]
   \begin{subfigure}{0.45\textwidth}
      % \includegraphics[width=\textwidth]{}
   \end{subfigure}
   \begin{subfigure}{0.45\textwidth}
      % \includegraphics[width=\textwidth]{}
   \end{subfigure}
   \caption{Results of original and optimized geometry in CST.}
\end{figure}
\end{center}

\subsection{Questions}
\begin{itemize}
   \item new maximal volume for electrode?
   \item volume constraint for anode ring?
   \item make size of patch 1/11 also a parameter?
   \item how much is inner insulator or already PEC?

   \item let constraint for Lagrangian return Inf or rather make all similar magnitude?
\end{itemize}

\subsection{Astra}
In the future we may also optimize the electrode boundary next to the puck, i.~e.~patches 3, 4 and 5 to obtain optimal particle trajectories. The cost function will be computed using Astra.
The desired total bunch charge is $10$ pC with a beam current of $(20-100)\ \mu\mathrm{A}$, whereas a typical value would be $100$ fC. The bunch length is around $5$ ps with a normalized transversal emittance of $e_{x,y} \leq 1\ \mathrm{mm\: mrad}$. The desired energy resolution is $\frac{\Delta E}{E} \leq 10^{-4}$.
The tracking will be performed using individual bunches from a pulsed laser. The emission model may be derived from \cite{wagner}.
